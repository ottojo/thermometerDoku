\documentclass{article}
\usepackage[ngerman]{babel}
\usepackage[autostyle=true,german=quotes]{csquotes}
\usepackage{siunitx}
\usepackage[european, siunitx]{circuitikz}
\usepackage{listings}
\usepackage{xcolor}
\usepackage{url}
\usepackage{IEEEtrantools}
\usepackage[utf8]{inputenc}



% set the default code style
\lstset{
    frame=tb, % draw a frame at the top and bottom of the code block
    tabsize=4, % tab space width
    showstringspaces=false, % don't mark spaces in strings
    numbers=left, % display line numbers on the left
    commentstyle=\color{green}, % comment color
    keywordstyle=\color{blue}, % keyword color
    stringstyle=\color{red} % string color
}

\title{Konstruktion eines Bluetooth-Thermometers}
\author{Jonas Otto}
\date{\today}

\begin{document}

\renewcommand\refname{Referenzen}
\bstctlcite{BSTcontrol} %Using IEEEtranBSTCTL in bib for translating

\maketitle
\newpage

\tableofcontents
\newpage

\section{Einleitung}

Im Rahmen dieses Projektes soll eine elektronisches Thermometer entwickelt
werden. Dieses soll die Temperatur messen und auf einem PC ausgeben. Alternativ
kann die Temperatur auch per Bluetooth an ein Smartphone geschickt und dort
ausgegeben werden.

\section{Elektronik}

Verwendet wird der Temperatursensor \enquote{PT100}, der bei einer Temperatur
von \SI{0}{\degreeCelsius} einen Widerstand von \SI{100}{\ohm} besitzt. Dieser
Widerstand wird mit Hilfe eines Spannungsteilers gemessen, die resultierende
Spannung $V_{st}$ wird mit einem Differenzverstärker auf Werte zwischen
\SI{0}{\volt} und \SI{5}{\volt} verstärkt. Diese Spannung $V_{adc}$ wird von
einem Analog-Digital-Converter, der in dem Atmega328P Microcontroller auf dem
Arduino UNO Board integriert ist, gemessen.

\subsection{Schaltplan}
\begin{center}
  \begin{circuitikz}
    \ctikzset{label/align = rotate}
    \draw
    (0,0) node[op amp] (opv1) {}
    %Non-Inverting input
    (-2, -.5) to (-2, 1) to[R=100<\ohm>] (-2,3) node[vcc] {5V} %Pullup
    (-2, -2.5) node[ground] {} to[thR=PT100, -*] (-2, -.5)    %PT100
    (opv1.+) to (-2, -.5)
    %Inverting input
    (-3,-2.5) node[ground] {} to[R=1<\kohm>] (-3, -.5) to[short, -*] (-3, .5) to (opv1.-)
    (-3, .5) to (-3, 1) to[R=1<\kohm>] (-3,3) node[vcc] {5V}
    %Output
    (opv1.out) node[right] {$V_{adc}$}
    (opv1.out) --++(1,0) node[right] {$V_{adc}$}
    %Power
    (opv1.up) --++(0,0.5) node[vcc]{\SI{5}{\volt}}
    (opv1.down) --++(0,-0.5) node[ground]{}
    %Feedback R
    (opv1.-) to[short,*-] (-1.2,2.5)
    (1.5,2.5) to[short,-*] (1.5,0)
    (-1.2,2.5) [R=4.7<\kohm>] to(1.5,2.5)
    ;
  \end{circuitikz}
\end{center}

\subsection{Layout}

Die Platine soll als Arduino-Shield hergestellt werden, also direkt auf das Arduino
UNO Entwicklungsboard aufsteckbar sein. Dafür wurde in der Software \enquote{Target 3001!}
eine Vorlage\cite{arduinoTarget} benutzt. Für eine einfache Herstellung wurde
eine einseitige Platine gewählt, welche zwar für Einschränkungen im Layout
sorgt, dafür aber einfach herzustellen ist. Die Bauteile befinden sich später
oben, die Leiterbahnen auf der Unterseite. Es werden ausschließlich
Through-Hole Komponenten verwendet, für den Operationsverstärker war ein DIL
Package verfügbar, welches gesockelt verbaut wird. Die Schwierigkeit lag bei
diesem Schritt darin, das Layout sowohl einfach und kompakt zu halten, als auch
einen einfachen Aufbau zu gewährleisten.

\subsection{Bluetooth}

Für die Bluetooth Kommunikation wird ein \enquote{HC-05} Modul eingesetzt,
welches bereits mit Elektronik zur Spannungsversorgung mit \SI{5}{\volt}
erhältlich ist. Da die serielle Schnittstelle des HC-05 auf einem Spannungslevel
von \SI{3.3}{\volt} arbeitet, der Arduino UNO aber auf \SI{5}{\volt}, wurde
ein Modul gewählt, welches auch Elektronik zum Level-Shifting enthält.


\section{Software}

Die Software besteht aus zwei Teilen: Auf dem Microcontroller wird die
Temperatur berechnet und per Bluetooth an verbundene Geräte geschickt. Eine
Android App zeigt die aktuelle Temperatur an und erlaubt die Konfiguration des
Thermometers.

\subsection{Microcontroller}

  \begin{lstlisting}[language=C++, caption={Arduino code}]
  #include <SoftwareSerial.h>

  #define bluetoothRxPin 10
  #define bluetoothTxPin 11
  #define tmpPin A0

  #define cFactor -16.0 //Temperature at 0V
  #define mFactor 0.36 //Volts per degree C

  SoftwareSerial bluetoothSerial(bluetoothRxPin, bluetoothTxPin);

  void setup() {
    bluetoothSerial.begin(9600);
    Serial.begin(9600);
  }

  void loop() {
    double tmpVoltage = analogRead(tmpPin) * (5.0/1023);
    double temperatureCelsius = tmpVoltage * (1.0/mFactor) - cFactor;

    Serial.print("Temperature: ");
    Serial.print(temperatureCelsius);
    Serial.println("C");

    bluetoothSerial.println(temperatureCelsius);

    delay(500);
  }
  \end{lstlisting}

  \subsubsection{Temperaturberechnung}
    Der Arduino UNO Microcontroller wird in einem angepassten C++ Dialekt
    programmiert. Die Hauptfunktion des Programmes ist die Umrechnung der gemessenen
    Spannung in eine Temperatur:
    \[
    \Theta = V_{t} * {1 \over m} - c
    \]
    $c$ ist dabei die Temperatur bei \SI{0}{\volt}, $m$ die mittlere Steigung der
    Spannung in Abhängigkeit zur Temperatur in $V \over \si{\degreeCelsius}$.

  \subsubsection{Bluetooth Kommunikation}
    Für die Bluetooth Kommunikation wird ein HC-05 Modul verwendet. Nach
    initialer Konfiguration können Daten mittels serieller Schnittstelle an das
    Smartphone übertragen werden. Die Temperatur wird ein Mal pro Sekunde als
    signed integer übertragen, gefolgt von einem newline Zeichen.

  \subsection{Smartphone}
    Zur Anzeige der Temperatur soll ein Smartphone dienen. Eine darauf laufende
    App soll die Temperatur mit entsprechender Einheit gut lesbar anzeigen und
    diese Anzeige regelmäßig aktualisieren. Zum jetztigen Zeitpunkt befindet
    sich diese App noch nicht in der Entwicklung, weshalb zu Testzwecken die App
    \enquote{Bluetooth Terminal} \cite{bluetoothTerminal} verwendet wird.

\bibliographystyle{IEEEtran}
\bibliography{doku}

\end{document}
